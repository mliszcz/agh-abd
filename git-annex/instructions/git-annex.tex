\documentclass[polish]{article}
\usepackage[T1]{fontenc}
\usepackage[utf8]{inputenc}
\usepackage[polish]{babel}
\usepackage{a4wide}
\usepackage{color}
\usepackage{latexsym}
%\usepackage[dvips]{graphicx}
%\usepackage[dvips]{epsfig}
%\usepackage{url}
\usepackage{listings}

\usepackage{hyperref}

\definecolor{uxgray}{gray}{0.75}
\newcommand{\uxcmd}[1]{\colorbox{uxgray}{\scalebox{0.6}[0.9]{\texttt{#1}}}}
\newcommand{\ipbox}[0]{\_\_\_.\_\_\_.\_\_\_.\_\_\_}
\newcommand{\tleft}[1]{
\begin{flushleft}
#1
\end{flushleft}
}

%\newenvironment {titemize}{\begin{itemize} \setlength{\itemsep}{-\parsep} } {\end{itemize}}

\title{\textbf{Instrukcja do ćwiczenia}\\git-annex}
\author{Michał Liszcz \and Jakub Sawicki}
\begin{document}
\maketitle

\begin{tabular}{|l|p{.7\textwidth}|}
\hline
Data wykonania & \\
\hline
Skład Grupy & \\
\hline
Ocena & \\
\hline
\end{tabular}

\vspace{0.5cm}
\noindent \textbf{Podczas wykonywania ćwiczenia odznaczaj wykonane podpunkty!}

\noindent Przed przystąpieniem do ćwiczenia sprawdź obecność i stan sprzętu.
Wszelkie nieprawidłowości należy natychmiast zgłosić prowadzącemu.
\vspace{0.5cm}

\renewcommand{\labelenumi}{$\Box$~\texttt{\theenumi}}
\renewcommand{\labelenumii}{$\Box$~\texttt{\theenumii}}

%\subsection*{Zestawienie sprzętu}

%W skład stanowiska wchodzą dwa komputery oraz biblioteka dysków
%magneto-optycznych HP 660ex. Biblioteka podpięta jest do komputera o
%nazwie droid2.


\begin{enumerate}

    \item
    Skonfiguruj dwie maszyny wirtualne \texttt{centos6} z zainstalowanymi
    natępującymi paczkami (git-annex dostępny jest w repozytorium EPEL,
    `yum install epel-release`, pozostałe są w standardowych repozytoriach):

    \begin{itemize}

      \item git >= 1.7.1
      \item rsync >= 3.0.6
      \item openssh-server >= 5.3p1
      \item git-annex >= 3.20120522
    \end{itemize}

    Dodaj w pliku \texttt{/etc/hosts} wpisy, by maszyny były dostępne pod
    nazwami \texttt{hostA} i \texttt{hostB}.

    \item
    Na obu maszynach skonfiguruj serwer SSH, utwórz użytkownika \texttt{git}
    oraz utwórz dla niego parę kluczy (\texttt{ssh-keygen}) Wymień klucze
    publiczne między maszynami (\texttt{ssh-copy-id}) tak, by było możliwe
    logowanie przy ich pomocy.

    Sprawdź czy z każdej maszyny możesz zalogować się na drugą z nich:

    \begin{lstlisting}
        git@hostA:~$ ssh git@hostB
        git@hostB:~$ ssh git@hostA
    \end{lstlisting}

    \item
    Utwórz na obu maszynach repozytoria \texttt{hostA-main} i
    \texttt{hostB-main} (gdzie prefiks oznacza maszynę, na której dane
    repozytorium powinno się znajdować). w każdym z repozytoriów zainicjalizuj
    git-annex i dodaj drugie repozytorium jako remote.

    \begin{lstlisting}
        git@hostA:~$ git init hostA-main.git
        git@hostA:~$ cd hostA-main.git
        git@hostA:hostA-main.git$ git annex init 'hostA - main'

        ...

        git@hostA:hostA-main.git$ git remote add \
            hostB-main git@hostB:hostB-main.git
    \end{lstlisting}

    \item
    W repozytorium \texttt{hostA-main} utwórz plik \texttt{important\_file.txt}
    i dodaj go do indeksu git-annex. Uzyskaj dostęp do zawartości pliku na
    maszynie \texttt{hostB}. Zweryfikuj gdzie znajduje się aktualnie plik z
    \texttt{hostA} i \texttt{hostB}.

    \item
    Zmodyfikuj zawartość pliku w \texttt{hostB-main} i zweryfikuj jak
    rozprzestrzeniają się zmiany.

    \item
    Utwórz kolejne repozytorium \texttt{hostB-media}. Może to być repozytorium,
    które przechowywane będzie na przykład na dysku backupowym
    \texttt{/dev/sdb}.

    \item
    Utwórz plik o wielkości 500MB w \texttt{hostB-main}. Zbadaj szybkość (czas)
    transferu przy kopiowaniu pliku do \texttt{hostA-main} (przez SSH) i do
    \texttt{hostB-media} (kopia lokalna).

    \begin{tabular}{|l|p{.7\textwidth}|}
    \hline
    SSH & \\
    \hline
    lokalnie & \\
    \hline
    \end{tabular}

    \item
    Zbadaj działanie flagi numcopies. Zwróć uwagę na to, że wersja git-annex
    dostępna w repozytorium centos6 nie implementuje jeszcze polecenia
    \texttt{numcopies}.

    \begin{itemize}
        \item Ustaw flagę \texttt{numcopies} na 2.
        \item Następnie utwórz w wybranym repozytorium bardzo ważny plik
              (\texttt{my\_data.txt}) i skopiuj go do jednego z pozostałych
              repozytoriów, tak by w systemie były dwie kopie.
        \item Spróbuj porzucić plik lokalnie i obserwuj wyniki.
    \end{itemize}

    \item
    Umieść plik z poprzedniego punktu w dwóch zdalnych repozytoriach i porzuć
    lokalną kopię. Spróbuj pobrać ją z flagą \texttt{get --auto}. Zasymuluj
    awarię jednego ze zdalnych repozytoriów a następnie usuń je z indeksu
    git-annex. Sprawdź ile kopii pliku ustnieje w systemie. Czy teraz możesz
    pobrać go do lokalnego repozytorium z użyciem flagi \texttt{--auto}?.

    \item
    \textbf{Zadanie dodatkowe.} Załóż konto w serwisie
    \href{https://gitlab.com}{https://gitlab.com}
    \footnote{Gitlab.com oferuje 10GB darmowego miejsca na przechowywanie
    plików z użyciem systemu git-annex}.
    Gitlab wspiera tylko autentykację przy użyciu klucza publicznego. Dodaj
    w panelu ustawień konta klucz wygenerowany na początku ćwiczenia.
    Utwórz repozytorium i dodaj je jako remote do istniejącego repozytorium.
    Zsynchronizuj status git-annex. Zbadaj szybkość transferu 50MB pliku do
    Gitlab.com, do drugiej maszyny w sieci lokalnej i do repozytorium na tej
    samej maszynie.

    \begin{tabular}{|l|p{.7\textwidth}|}
    \hline
    SSH & \\
    \hline
    lokalnie & \\
    \hline
    Gitlab.com & \\
    \hline
    \end{tabular}

\end{enumerate}

\end{document}
