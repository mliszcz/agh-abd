\documentclass[polish]{article}
\usepackage[T1]{fontenc}
\usepackage[utf8]{inputenc}
\usepackage[polish]{babel}
\usepackage{a4wide}
\usepackage{color}
\usepackage{latexsym}
%\usepackage[dvips]{graphicx}
%\usepackage[dvips]{epsfig} 
%\usepackage{url} 
\usepackage{listings}

\definecolor{uxgray}{gray}{0.75}
\newcommand{\uxcmd}[1]{\colorbox{uxgray}{\scalebox{0.6}[0.9]{\texttt{#1}}}}
\newcommand{\ipbox}[0]{\_\_\_.\_\_\_.\_\_\_.\_\_\_}
\newcommand{\tleft}[1]{
\begin{flushleft} 
#1
\end{flushleft}
}

%\newenvironment {titemize}{\begin{itemize} \setlength{\itemsep}{-\parsep} } {\end{itemize}} 

\title{\textbf{Instrukcja do ćwiczenia}\\git-annex}
\author{Michał Liszcz \and Jakub Sawicki}
\begin{document}
\maketitle

\begin{tabular}{|l|p{.7\textwidth}|}
\hline
Data wykonania & \\
\hline
Skład Grupy & \\
\hline
Ocena & \\
\hline
\end{tabular}

\vspace{0.5cm}
\noindent \textbf{Podczas wykonywania ćwiczenia odznaczaj wykonane podpunkty!}

\noindent Przed przystąpieniem do ćwiczenia sprawdź obecność i stan sprzętu.
Wszelkie nieprawidłowości należy natychmiast zgłosić prowadzącemu.
\vspace{0.5cm}

\renewcommand{\labelenumi}{$\Box$~\texttt{\theenumi}}
\renewcommand{\labelenumii}{$\Box$~\texttt{\theenumii}}

%\subsection*{Zestawienie sprzętu}

%W skład stanowiska wchodzą dwa komputery oraz biblioteka dysków
%magneto-optycznych HP 660ex. Biblioteka podpięta jest do komputera o
%nazwie droid2.


\begin{enumerate}
    \item Skonfiguruj dwie maszyny wirtualne centos6 z zainstalowanymi odpowiednimi
        paczkami.
        Maszyny niech będą dostępne dla siebie pod nazwami \texttt{hostA} i \texttt{hostB}.
        \begin{lstlisting}
docker-compose up
docker-compose exec --user git hostA /bin/bash
git@hostA:~$ ssh-keygen
git@hostA:~$ cat 'git' | ssh-copy-id git@hostB
docker-compose exec --user git hostB /bin/bash
git@hostB:~$ ssh-keygen
git@hostB:~$ cat 'git' | ssh-copy-id git@hostA
        \end{lstlisting}

\   \item Utwórz na obu maszynach repozytoria \texttt{hostA-main} i \texttt{hostB-main} (gdzie
        prefiks oznacza maszynę, na której dane repozytorium powinno się znajdować).
        Remote'y maszyn powinny nawzajem na siebie wskazywać.

        \begin{lstlisting}
git@hostA:~$ git init hostA-main.git
git@hostA:~$ cd hostA-main.git
git@hostA:hostA-main.git$ git annex init 'hostA - main'

git@hostB:~$ git init hostB-main.git
git@hostB:~$ cd hostB-main.git
git@hostB:hostB-main.git$ git annex init 'hostB - main'
git@hostB:hostB-main.git$ git remote add hostA-main git@hostA:hostA-main.git
git@hostB:hostB-main.git$ git pull hostA-main master # nie zadziala dopoki nie ma commitow na master

git@hostA:hostA-main.git$ git remote add hostB-main git@hostB:hostB-main.git
        \end{lstlisting}

    \item W repozytorium hostA-main utwórz plik \texttt{important\_file.txt} i dodaj go do
   indeksu git-annex.
   Uzyskaj dostęp do zawartości pliku na maszynie \texttt{hostB}.
   Zweryfikuj gdzie znajduje się aktualnie plik z \texttt{hostA} i \texttt{hostB}.

   \begin{lstlisting}
git@hostA:hostA-main.git$ echo "important big file" > important_file.txt
git@hostA:hostA-main.git$ git annex add important_file.txt
git@hostA:hostA-main.git$ git commit -m "important_file"

git@hostB:hostB-main.git$ git annex sync
git@hostB:hostB-main.git$ ll
total 4
lrwxrwxrwx 1 git git 178 May 19 10:18 important_file.txt -> .git/annex/objects/3J/Z8/SHA256-s19--ac2b10210a29d0275fec27792ecbc2b641669cfc76b0a36b2a7062de274ecf31/SHA256-s19--ac2b10210a29d0275fec27792ecbc2b641669cfc76b0a36b2a7062de274ecf31
git@hostB:hostB-main.git$ cat important_file.txt
cat: important_file.txt: No such file or directory
git@hostB:hostB-main.git$ git annex get important_file.txt
git@hostB:hostB-main.git$ cat important_file.txt
important big file
git@hostB:hostB-main.git$ git annex whereis important_file.txt
whereis important_file.txt (2 copies)
    bad498fb-e968-467a-98a8-c59fd7800932 -- hostA-main (hostA - main)
    f0c9f4ba-7550-40ce-9eb0-a325b991d306 -- here (hostB - main)
ok

git@hostA:hostA-main.git$ git annex whereis important_file.txt
whereis important_file.txt (1 copy)
    bad498fb-e968-467a-98a8-c59fd7800932 -- here (hostA - main)
ok
git@hostA:hostA-main.git$ git annex sync
git@hostA:hostA-main.git$ git annex whereis important_file.txt
whereis important_file.txt (2 copies)
    bad498fb-e968-467a-98a8-c59fd7800932 -- here (hostA - main)
    f0c9f4ba-7550-40ce-9eb0-a325b991d306 -- hostB-main (hostB - main)
ok
   \end{lstlisting}

\item Zmodyfikuj zawartość pliku w \texttt{hostB-main} i zweryfikuj jak rozprzestrzeniają
   się zmiany.

   \begin{lstlisting}
git@hostB:hostB-main.git$ echo " modified" >> important_file.txt
bash: important_file.txt: Permission denied
git@hostB:hostB-main.git$ git annex unlock important_file.txt
git@hostB:hostB-main.git$ ll
total 4
-rw-r--r-- 1 git git 20 May 19 10:55 important_file.txt
git@hostB:hostB-main.git$ echo " modified" >> important_file.txt
git@hostB:hostB-main.git$ git commit important_file.txt -m "Modified important_file.txt"
git@hostB:hostB-main.git$ ll
total 4
lrwxrwxrwx 1 git git 178 May 19 10:55 important_file.txt -> .git/annex/objects/wG/2j/SHA256-s29--ac1d39fe5ee700d3b1bd2b464fd4dc47b847a4e8dd6d796451ba5ca836cce2e3/SHA256-s29--ac1d39fe5ee700d3b1bd2b464fd4dc47b847a4e8dd6d796451ba5ca836cce2e3
git@hostB:hostB-main.git$ git annex whereis important_file.txt
whereis important_file.txt (1 copy)
    f0c9f4ba-7550-40ce-9eb0-a325b991d306 -- here (hostB - main)
ok

git@hostA:hostA-main.git$ git annex whereis important_file.txt
whereis important_file.txt (2 copies)
    bad498fb-e968-467a-98a8-c59fd7800932 -- here (hostA - main)
    f0c9f4ba-7550-40ce-9eb0-a325b991d306 -- hostB-main (hostB - main)
ok
git@hostA:hostA-main.git$ git annex sync
git@hostA:hostA-main.git$ git annex whereis important_file.txt
whereis important_file.txt (1 copy)
    f0c9f4ba-7550-40ce-9eb0-a325b991d306 -- hostB-main (hostB - main)
ok
   \end{lstlisting}

\item Utwórz kolejne repozytorium \texttt{hostB-media}.
   Może to być repozytorium, które przechowywane będzie na przykład na napędzie USB.

   \begin{lstlisting}
git@hostB:~$ git init hostB-media.git
git@hostB:~$ cd hostB-media.git
git@hostB:hostB-media.git$ git annex init 'hostB - media'
git@hostB:hostB-media.git$ git remote add hostB-main /home/git/hostB-main.git
git@hostB:hostB-media.git$ git pull hostB-main master
git@hostB:hostB-media.git$ git annex sync

git@hostB:hostB-main.git$ git remote add hostB-media /home/git/hostB-media.git
git@hostB:hostB-main.git$ git annex sync
   \end{lstlisting}

\item Utwórz plik o wielkości 500MB w hostB-main.
   Zbadaj szybkość transferu do hostA-main i hostB-media.

   \begin{lstlisting}
git@hostB:hostB-main.git$ dd if=/dev/zero of=large.bin bs=1024 count=500000
git@hostB:hostB-main.git$ git annex add large.bin
git@hostB:hostB-main.git$ git commit -m "Added large.bin"
git@hostB:hostB-main.git$ git annex copy large.bin hostB-media
copy large.bin (to hostB-media...) ok
(Recording state in git...)
git@hostB:hostB-main.git$ git annex copy large.bin hostA-main
copy large.bin (checking hostA-main...) (to hostA-main...)
SHA256-s512000000--4c98cf638799a07eb85872e7f5f5d8d661c09e6e15af02f3655ed9250cae13b0
   512000000 100%   83.22MB/s    0:00:05 (xfer#1, to-check=0/1)

sent 512062644 bytes  received 31 bytes  78778873.08 bytes/sec
total size is 512000000  speedup is 1.00
ok
   \end{lstlisting}

\item Zbadaj działanie flagi numcopies.
   Zwróć uwagę na to, że wersja git-annex dostępna w repozytorium centos6 nie
   implementuje jeszcze polecenia \texttt{numcopies}.

   \begin{lstlisting}
git@hostB:hostB-main.git$ git config annex.numcopies 2 # konfig sie nie propaguje przy sync-u
git@hostB:hostB-main.git$ git annex copy important_file.txt --to hostA-main
git@hostB:hostB-main.git$ git annex drop important_file.txt
drop important_file.txt (checking hostA-main...) (unsafe)
  Could only verify the existence of 1 out of 2 necessary copies

  No other repository is known to contain the file.

  (Use --force to override this check, or adjust annex.numcopies.)
failed
git-annex: drop: 1 failed
   \end{lstlisting}

\end{enumerate}



\end{document}
